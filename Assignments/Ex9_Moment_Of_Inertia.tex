\documentclass[12pt,letterpaper,fleqn]{article}
\setlength\parindent{0pt}
%Packages
\usepackage[total={7in,10in},top=0.25in,left=0.75in,includehead]{geometry}
% \usepackage{fancyhdr}
% \pagestyle{fancy}
\usepackage{graphicx}
\usepackage{calc}
%\usepackage{ascmac}

\begin{document}
\framebox{\begin{minipage}[t]{\widthof{ Austin Shelton }}
Lab Partner\\
Angel Avina\\
Austin Shelton
\end{minipage}}\hfill
\framebox{\begin{minipage}[t]{\widthof{ November 12th 2013 }}
Kaya Ota\\
Mohsen Janatpour\\ 
PHYS250\\
November 12th 2013
\end{minipage}}\hfill
%\vspace{2ex}
\begin{center}
{\LARGE\bf Experiment 9\linebreak  
            Moment of Inertia}
\end{center}



\begin{flushleft}
{\LARGE\bf Purpose}\\
{\hspace{3pt}This experiment is to compare the experimental moment of inertia of disk and ring with the theoretical moment of inertia of each.    }
\end{flushleft}

\begin{flushleft}
{\LARGE\bf Procedure}
\begin{enumerate}
\item Set up the tools to be leveled and measure weights and diameters of the ring and the disk.
\item Find friction mass to determine how much hanging mass can hold the system to be rest. 
\item After finding friction masses, replace by experimental mass at the useful height and measure the height from the ground.
\item measure the elapsed time that the mass reach to the ground.
\item execute three times and change the conditions of system little. 
\end{enumerate}
\end{flushleft}



\begin{flushleft}
{\LARGE\bf Data}
\end{flushleft}

\begin{center}
{\large\bf Measured Data}
\end{center}

\begin{table}[htb]
\begin{center}
	\begin{tabular}{|r||c||}\hline
	Name of Materials & Values[unit]\\\hline\hline
    Mass of Ring & $1422.3[g]$\\\hline
    Mass of Disk & $ 1417.2[g]$ \\\hline\hline
    Diameter of Disk  &$22.8[cm] $\\\hline
    Diameter of Outer Ring & $12.6[cm]$ \\\hline
    Diameter of Inner Ring &$10.6[cm] $   \\\hline 
    Diameter of Drum & $2.58[cm] $\\\hline\hline
    Friction Mass of Disk & $3[g]$\\\hline
    Friction Mass of Disk With Ring & $5[g]$\\\hline\hline
    Height Between System and the Ground & $91.1[cm]$\\\hline
    \end{tabular}
\end{center}
\end{table}
\newpage
%%%%%%%%%%%%%%%%%%%%%%%%%%%%%%%%%%%%%%%%%%%%%%%%%%%%%%%%%%%%%%%%%%%%%%%%
\begin{center}
{\large\bf Run Time Without Ring (Horizontally)}
\\ Hanging mass is 50[g]
\end{center}

\begin{table}[htb]
\begin{center}
%data for the case set up only disk horizontally 
\begin{tabular}{|r||c|}\hline
	1st run & 13.96[sec]\\\hline 
	2nd run & 14.18[sec]\\\hline
	3rd run &14.43[sec]\\\hline
\end{tabular}
\end{center}
\end{table}

\begin{center}
{\large\bf Run Time With Ring (Horizontally)}
\\ Hanging mass is 100[g]
\end{center}

\begin{table}[htb]
\begin{center}
%data for the case put ring on the disk setting horiontally
\begin{tabular}{|r||c|}\hline
	1st run & 12.50[sec]\\\hline 
	2nd run & 12.97[sec]\\\hline
	3rd run & 12.53[sec]\\\hline
\end{tabular}
\end{center}
\end{table}



\begin{center}
{\large\bf Run Time Vertically}
\\ Hanging mass is 50[g] and friction mass is 3[g]\\
\end{center}


%data for the case setup vertically 
\begin{table}[htb]
\begin{center}
\begin{tabular}{|r||c|}\hline
	1st run & 10.55[sec]\\\hline
	2nd run & 10.53[sec]\\\hline
	3rd run & 10.53[sec]\\\hline
\end{tabular}
\end{center}
\end{table}



\begin{flushleft}
{\large \bf Calculation}
\end{flushleft}

\begin{enumerate}
\item Calculate the speed and the transnational kinetic energy of the driving mass just as it strikes. 




\item
\item
\item
\end{enumerate}







































\end{document}