\documentclass[12pt,letterpaper,fleqn]{article}
\setlength\parindent{0pt}
%Packages
\usepackage[total={7in,10in},top=0.25in,left=0.75in,includehead]{geometry}
\usepackage{color}
%\pagestyle{fancy}
%\usepackage{fancybox}
\usepackage{graphicx}
\usepackage{calc}
%\usepackage{ascmac}

\begin{document}
\begin{center}
\textcolor{green}{\LARGE\bf Transnational Motion \& Rotational Motion  }
\end{center}

\begin{table}[htb]
\begin{center}
	\begin{tabular}{|r||c|c|}\hline
	Value & Transnational(Linear) Motion & Rotational(Angular) Motion \\\hline\hline
    Position{\tiny as a component} & $x(t) = x_0 + v_0t + \frac{1}{2}at^2 $ &
     $\theta(t) = \theta_0 + \omega_0t + \frac{1}{2} \alpha t^2$ \\\hline
    Velocity{\tiny as a component} & $ v(t) = v_0 + at $ & 
    $\omega(t) = \omega_0 + \alpha t $\\\hline
    Acceleration{\tiny as a component} & $ a $ & $\alpha \ \ \  a_{\parallel} = r\alpha \ \ \  a_{\bot} = r\omega^2 $\\\hline
    Mass & m & $ I = mr^2 $\\\hline
    Force & $\vec{F} = m\vec{a}$ & $\vec{\tau} = I\vec{\omega}$ \\\hline
    Work & $ W = \int_{x_1}^{x_2} \vec{F}\cdot d\vec{x}$ & $ W =\int_{\theta_1}^{\theta_2} \vec{\tau} \cdot d\vec{\theta}$\\\hline
    Kinetic Energy & $K_x = \frac{1}{2}mv^2 $ & $K_{\theta} = \frac{1}{2}I\omega^2 $\\\hline
    Momenta & $\vec{P} = m\vec{v} $ & $\vec{L} = I\vec{\omega} $\\\hline
    Impulse & $\vec{J} = \int_{t_1}^{t_2} \vec{F}dt $ & $ \vec{J_\theta} = \int_{t_1}^{t_2} \vec{\tau}dt$ \\\hline
    Net Impulse & $\vec{J_{net}} = \Delta \vec{P}$& $\vec{J_{\theta}} = \Delta \vec{L}$\\\hline
	\end{tabular} 
\end{center}
\end{table}

\begin{flushleft}
\textcolor{cyan}{\large\bf Definitions} 
\end{flushleft}
\begin{flushleft}

\underline{\large\it Line of Action}\\ 
\ \ \ \ \ the line along which the force vector lies.

\underline{\large\it Moment of Arm or Leverage}\\
\ \ \ \ \ Distance from the pivot potions to the line of action.


\underline{{\large\it Torque} \tiny$\leftarrow$  this guy actually causes "rotation"}\\
\ \ \ \ \ Vector of Torque:\\
\ \ \ \ \ \ \ \ \ \ \ $\vec{\tau} = \vec{r}\times\vec{F}$\\
\ \ \ \ \ Magnitude of Torque: 
\begin{eqnarray}
\ \ \ \ \  \tau &=& Fl \nonumber \\
&=& rF\sin(\phi) \nonumber \\
&=& F_{tan}r \nonumber \\
\end{eqnarray} 


\end{flushleft}






%\begin{enumerate}
%\item [{\large \bf Moment of Arm OR Leverage}] Distance from the pivot point to the line of action 
%\end{enumerate}

\end{document}