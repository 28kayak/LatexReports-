\documentclass[12pt,letterpaper,fleqn]{article}
\setlength\parindent{0pt}
%Packages
\usepackage[total={7in,10in},top=0.25in,left=0.75in,includehead]{geometry}
\usepackage{color}
%\pagestyle{fancy}
%\usepackage{fancybox}
\usepackage{graphicx}
\usepackage{calc}
%\usepackage{ascmac}

\begin{document}
\framebox{\begin{minipage}[ht]{\widthof{ November 12th 2013 }}
Kaya Ota\\
Mohsen Janatpour\\ 
PHYS250\\
November 12th 2013
\end{minipage}}\hfill
\begin{center}
{\LARGE\bf Pole vault \& Kinetic Energy}
\end{center}
{\large\bf aaa}\\
\ \ \ \ \ \ Question comes from what matters when people try to jump using pole.
What I was thinking was skinnier people can jump higher, maybe.
However, according to the conservation of energy theorem, it only mattes how fast you can run because variable m indicating mass is canceled out during calculation just like below.

Reminded, the conservation of energy theorem is:
\begin{eqnarray}
Kinetic Energy &=& Potential Energy\\
\frac{1}{2} mv^2 &=& mgh\\
\frac{1}{2} v^2 &=& gh\\
h &=& \frac{1}{2g} v^2 \\ 
\end{eqnarray}

Now, it is obvious that the height, h human could jump totally depends on how fast the human can run because a speed v is only the variable here.
This claims us who can jump the highest in the world is amazingly equals to the person who can run the fastest in the world.
Check out how high the fastest human can jump theoretically.\\
{\large \bf Data from the Internet}

\begin{table}[htb]
\begin{center}
\begin{tabular}{|r||c|c||}\hline
Fastest human & $44[\frac{km}{h}]$ & $17.2 [\frac{m}{sec}]$\\\hline

\end{tabular}
\end{center}
\end{table}


 






\end{document}


